\begin{abstract}
  Reinforcement learning буюу бататгасан сургалтад тасралттай үйлдлийн хувьд суралцах үйл явц нь санамсаргүй үйлдлийг сонгох замаар явагддаг. Харин үргэлжилсэн үйлдлийн хувьд суралцах үйл явц нь үйлдэлд шуугианыг нэмэх замаар явагддаг. Deep Deterministic Policy Gradient (DDPG) бол тасралтгүй, үргэлжилсэн үйлдлүүдийг сурахад зориулагдсан алгоритм тул action space noise буюу үйлдлийн шуугиан ашиглагдана. Энэ судалгааны ажлаар 3 төрлийн шуугианыг харьцуулах, турших бөгөөд ямар үр нөлөөтэй, аль шуугиан нь илүү болохыг тодорхойлно.
  
\section{Зорилго}

DDPG бататгасан сургалтын үйлдлийн шуугианыг туршин, харьцуулж энэ шуугиан моделийг сургах үйл явцад хэрхэн нөлөөлж буйг ажиглан дүгнэлт гаргах

\section{Зорилтууд}

\begin{itemize}
	\item Моделийг DDPG алгоритмыг ашиглан BiPedalWalker-V3 орчныг дуустал алхаж чаддаг болгох
	\item Бие биенээсээ хамааралтай шуугиан, бие биенээсээ хамааралгүй шуугиан, параметр шуугиан гэх гурван төрлийн шуугианыг амжилттай хэрэгжүүлэх
	\item Хэрэгжүүлэлт хийхдээ кодыг аль болох ойлгомжтой DDPG алгоритын pseudo кодын дагуу хэрэгжүүлэх
\end{itemize}
  
\end{abstract}
