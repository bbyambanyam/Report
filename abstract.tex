\begin{abstract}

Сүүлийн жилүүдэд бататгасан сургалтын (RL) алгоритмуудыг компьютерын шинжлэх ухааны судлаачид ихээр сонирхон судалж байна. Ялангуяа Model-free RL алгоритмуудыг олон янзын төвөгтэй асуудлуудыг маш өндөр үр дүнтэйгээр шийдвэрлэхэд ихээр ашиглаж байгаа бөгөөд гүн неорал сүлжээг (DNN) нэмж ашигласнаар үр дүнг илүү сайжруулж байна.

RL алгоритмуудыг судлах, үнэлэх зориулалттай олон төрлийн сангууд бий болсоор байна. Эдгээрийн нэг нь OpenAI бөгөөд энэ нь нээлттэй эх сурвалж бүхий сан юм. Үүнд Gym буюу бататгасан сургалтын алгоритмуудыг хөгжүүлэх, харьцуулахад зориулсан хэрэгсэл, Baseline сан буюу сүүлийн үеийн бататгасан сургалтын алгоритмуудыг Tensorflow ашиглан гүйцэтгэсэн хэрэгжүүлэлт бүхий сан зэрэг багтана. Deep Deterministic Policy Gradient (DDPG) бол Baseline санд хэрэгжүүлэгдсэн алгоритмуудын нэг бөгөөд олон янзын нарийн төвөгтэй даалгавруудыг үр дүнтэй шийдвэрлэж чадах алгоритм юм.

Бататгасан сургалтад тасралттай үйлдлийн хувьд суралцах үйл явц нь санамсаргүй үйлдлийг сонгох замаар явагддаг. Харин үргэлжилсэн үйлдлийн хувьд суралцах үйл явц нь үйлдэлд шуугианыг нэмэх замаар явагддаг. DDPG бол тасралтгүй, үргэлжилсэн үйлдлүүдийг сурахад зориулагдсан алгоритм тул шуугиан ашиглагдана. Энэ судалгааны ажлаар 3 төрлийн шуугианыг харьцуулах, турших бөгөөд ямар үр нөлөөтэй, аль шуугиан нь илүү болохыг тодорхойлно.
  
\section*{Зорилго}

DDPG бататгасан сургалтын үйлдлийн шуугианыг туршин, харьцуулж энэ шуугиан моделийг сургах үйл явцад хэрхэн нөлөөлж буйг ажиглан дүгнэлт гаргах

\section*{Асуудал}

PyTorch санг ашиглан хэрэгжүүлсэн DDPG алгоритмын үр дүн ямар төрлийн шуугиан ашиглахаас хамааран хэрхэн өөрчлөгдөж байна вэ? Аль шуугиан нь илүү үр дүнтэй байна вэ?

\section*{Хамрах хүрээ}

Орчны шуугиануудыг хооронд нь харьцуулж, үнэлэхийн тулд ижил орчинд DDPG алгоритмыг ашиглан хэд хэдэн удаа туршин үзнэ. Энэхүү судалгааны ажлын бүх туршилтыг BipedalWalker-V3 симуляцийн орчинд хийнэ. Мөн бүх шуугианы хувьд mu=0, sigma=0.2 байна. Mu нь дундаж утга, sigma нь савлах утга. [-0.2,0.2] гэсэн хязгаарын дотор шуугианы утгыг авна гэсэн үг.
  
\end{abstract}
